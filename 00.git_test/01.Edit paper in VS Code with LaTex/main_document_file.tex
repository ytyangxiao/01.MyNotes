%-*- coding:UTF-8-*-
% gougu.tex
% 勾股定理

% preamble region
\documentclass[UTF8]{ctexart}
\usepackage{graphicx}
\usepackage{float}

\title{杂谈勾股定理}
\author{张三}
\date{\today}
\newtheorem{thm}{定理}
\bibliographystyle{plain}

% document body
\begin{document}

\maketitle
\begin{abstract}
这是一篇关于勾股定理的小短文。
\end{abstract}
\tableofcontents
\section{勾股定理在古代}
西方称勾股定理为毕达哥拉斯定理,将勾股定理的发现归功于公元前 6 世纪的毕达哥拉斯学派\cite{[18.1|1T]D}。该学派得到了一个法则,可以求出可排成直角三角形三边的三元数组。毕达哥拉斯学派没有书面著作,该定理的严格表述和证明则见于欧几里得\footnote{欧几里得,约公元前330-275年。}《几何原本》的命题 47: “直角三角形斜边上的正方形等于两直角边上的两个正方形之和。” 证明是用面积做出的。

我国《周髀算经》记载商高(约公元前 12 世纪)答周公问:
\begin{quote}
\zihao{-5}\kaishu	勾广三,股修四,径隅五。
\end{quote}

又载陈子(约公元前7--6世纪)答荣方问:
\begin{quote}
若求邪至日者,以日下为勾,日高为股,勾股各自乘,并而开方除之,得邪至日。
\end{quote}
都较古希腊更早。后者已经明确道出勾股定理的一般形式。图1是我国古代对勾股定理的一种证明。
\begin{figure}[ht]
	\centering
	\includegraphics[scale=0.6]{Fig_GouguDingLi.pdf}
	\caption{宋赵爽在《周髀算经》注中作的弦图,该图给出了勾股定理的一个极具对称美的证明。}
	\label{Fig_GouguDingLi}
\end{figure}

\section{勾股定理的近代形式}
勾股定理可以用现代语言表述如下:

\begin{thm}[勾股定理]
直角三角形斜边的平方等于腰的平方和。

可以用符号语言表述为:设直角三角形ABC,其中$\angle C=90^\circ$ ,则有

$AB^2=BC^2+AC^2$

\end{thm}

满足式(1)的整数成为\emph{勾股数}。第1节所说毕达哥拉斯学派得到的三元数组就是勾股数。下表列出一些较小的勾股数:
\begin{table}[H]
\begin{tabular}{|rrr|}
\hline
直角边$a$ & 直角边$b$ & 斜边$c$\\
\hline
3 & 4 & 5\\
5 & 12 & 13\\
\hline
\end{tabular}
\end{table}


\bibliography{Reference_file}

\end{document}